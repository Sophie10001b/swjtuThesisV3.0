%---------------------------------------------------------------------------------
%                西南交通大学研究生学位论文基本信息
%---------------------------------------------------------------------------------

% 请注意:本文件需用户填写论文的相应信息。

% 论文信息
% 定义申请学位
% 请注意:博士学位候选人输入 Doctor,硕士学位候选人输入 Master
\def\degree{Doctor} 

% 论文中文标题
% 请注意:标题控制在36个汉字以内),其中用 \\ 实现标题换行,单行不得超过18个汉字。
% 		  \underline命令用于实现标题的下划线,请把标题输入在\underline命令的的{}中
%		  V2.0, \underline{}已不再使用。
\cTitle{swjtuThesis V3.0}

% 论文英文标题(72个字符以内,包含空格)
\eTitle{swjtuThesis V3.0}


% 国内图书分类号
%\CI{CLS no.}
\CI{TP181, TP311}
% 国际图书分类号
%\UDC{UNC no.}
\UDC{004.8}
% 保密等级
\secLevel{公开}
% 单位代码
\UCODE{10613}

%----------------------------
% 学生信息
% 中英文姓名
\cAuthor{\#\quad \#} % 盲审使用 \author{\#\#\#}
\eAuthor{\#\#\#} % 盲审使用 \eAuthor{\#\#\#}

% 专业(10个汉字以内)
% 请注意:对于超过10个汉字的专业,比如“防灾减灾工程及防护工程”,目前排版仍然不够美观,正在修正。
%        也请相应专业的同学提供过往师兄的硕士论文以供参考。
\cDiscipline{计算机科学与技术}
\eDisciplineFirst{Computer Science and} % 第一行
\eDisciplineSecond{Technology} % 第二行,可为空

% 学号
\studentID{1111223333}

% 导师(10个汉字以内)
% 请注意:导师姓名和导师的职称之间加上\hspace{0.1em}以增大间距,整体显得更为美观。 
%\cTutor{导师\hspace{0.1em} 教授/博导}
%\eTutor{Professor Supervisor Name}
\cTutor{***} % \cTutor{***\hspace{0.1em} 教授}, 盲审使用-> \cTutor{***}
\eTutor{***} % \eTutor{Professor ***}, 盲审使用-> \eTutor{***}

% 学院
\cSchool{计算机与人工智能学院}
\eSchoolFirst{School of Computing and} % 第一行
\eSchoolSecond{Artificial Intelligence} % 第二行,可为空

% 封面日期,依次为年,月,日
\cDate{2024}{7}{4}
\eDate{2024}{Jul.}{4} % 例如,{2024}{May}