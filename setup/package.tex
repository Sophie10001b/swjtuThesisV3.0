%---------------------------------------------------------------------------------
%                学位论文LaTeX模板使用宏包文件
%---------------------------------------------------------------------------------

% 数学公式常用宏包(实现更多样的数学公式输入)
\usepackage{amsmath}
\usepackage{amssymb} % AMSLaTeX宏包, 用来排出更加漂亮的公式, 可以使用\eqref{*}而非(ref{*})来引用公式
\usepackage{mathrsfs} % 不同于\mathcal or \mathfrak 之类的英文花体字体
\usepackage{mathtools}
\usepackage{wasysym}
\usepackage{cases} % cases环境
%\usepackage{theorem}
\usepackage[amsmath,thmmarks]{ntheorem} % 定理类环境宏包,其中 amsmath 选项, 用来兼容 AMS LaTeX 的宏包. 与theorem不能同时使用.

% 算法宏包
\usepackage{algorithmic}
\usepackage[lined,boxed,ruled,algochapter]{algorithm2e} % algochapter算法按章编号
\renewcommand{\algorithmcfname}{算法}

% 表格制作常用宏包
\usepackage{multirow}

% 插入图片常用宏包(实现多种格式的调用)
\usepackage{graphicx}
\usepackage{subfig}

% 图表题注格式宏包(实现《规范》要求的题注格式)
% V3.0更改——参考兰大等学校的已发表毕业论文格式,多行图表题的行距变更为1.2倍
\usepackage{caption}
\captionsetup{labelformat=simple, labelsep=quad, labelfont=bf, textfont=normalfont, font={stretch=1.2}}

% TeX系列标识的正确输入宏包(实现正确插入TeX相关的字符串)
\usepackage{dtk-logos}

% 列表制作常用宏包(用于调用小间距列表)
\usepackage{paralist}

% 外框宏包
\usepackage{framed}

% 引用文献宏包natbib
%               上标	方括号	逗号分隔	自动排序分类
\RequirePackage[super,square,comma,sort&compress]{natbib}

% 颜色控制宏包
\usepackage{color,xcolor}
\usepackage{colortbl} %控制表格颜色

% 列表宏包
% V3.0更改——参考已发表论文格式,调整列表元素的缩进选项,达到首行缩进2字符的效果,同时去除了列表与正文的额外间隔
\usepackage{enumerate} % 其后可接选项[a,A,i,I,1]
\usepackage{enumitem}
\setenumerate[1]{itemsep=0pt,partopsep=0pt,parsep=\parskip,topsep=0pt,leftmargin=0pt,itemindent=3.5em}
\setitemize[1]{itemsep=0pt,partopsep=0pt,parsep=\parskip,topsep=0pt,leftmargin=0pt,itemindent=3.5em}
\setdescription{itemsep=0pt,partopsep=0pt,parsep=\parskip,topsep=0pt,leftmargin=0pt,itemindent=3.5em}

% 英文采用Times New Rome字体(建议把这个宏包放在最后避免发生宏包冲突)
\usepackage{fontspec}
\setmainfont{Times New Roman}
% 不需要设置CJKmainfont,ctex 宏包已经很好的处理了
% 不仅设置了粗体为黑体,斜体为楷体,还兼容了winfonts和adobefonts
